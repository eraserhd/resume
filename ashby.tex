\def\company{Ashby}
\relevantskill{IaC}{nexus,centralpark,coresystems}
\relevantskill{Kubernetes}{nexus,centralpark,coresystems}
\relevantskill{Data Warehouse}{homepage,nexus,coresystems}
\relevantskill{Security Work}{auth21,xpay,nexus}
\relevantskill{Scalability}{points,homepage,nexus}
\relevantskill{Canary+Gradual Rollouts}{points,homepage,finch}
\relevantskill{Event-Driven Architecture}{points}
\def\coverbody{
   Hello!

   I'm a Principal Software Engineer looking for my next gig.  While I've
   been a high-performing software engineer for ugh--too long to admit now,
   I've always been infrastructure-adjacent since building ISP infrastructure
   in 2000 and earning my RHCE in 2005.

   So I'm applying to a "Platform Engineer" position, not to take a back seat
   on software engineering, but because your challenges look interesting.
   "DSL-to-SQL compiler", eh?  I've always gravitated toward building services
   that help the entire org, and this often is infrastructure-related.

   Examples:

   One of my roles at 2U involved migrating 60+ Go services that built manually from a
   monorepo and deployed manually to a single EC2 instance.  I moved them into a company-blessed
   Kubernetes cluster, CI/CD pipeline, and Datadog.  This was a challenge
   because of resistance from the authoring contractors, but I succeeded through patience,
   advocacy and a bit of just letting the contractors look silly.  The lack of
   cooperation required me to dig in and understand the whole system, as many
   components as originally engineered couldn't be horizontally scaled or deployed
   in a zero-downtime fashion.  A lot of helm chart work was done here.

   Another of my roles at 2U required fixing performance on a Salesforce data-cache whose intent
   was to provide a single API unifying data across 20+ Salesforce instances.
   This ended up being a combination RDS sizing and schema refactoring to reduce
   the data size (for about 6 billion rows plus indexes) so that a high-memory RDS
   could cache the full main table without thrashing the LRU cache.
   Code work was necessary to use the new schema, and in-between dual-schema
   states were managed so the system had zero downtime while the data migration
   happened.

   Interestingly, this last project had a GraphQL-to-SQL compiler itself.
   It supported conjunction of multiple terms and value-set membership operators,
   and I completed out the features by adding disjunction, grouping, and
   negation, per user requests.  A different project that I wrote had a full
   GraphQL-to-Datalog compiler with all the relational and set operations and
   negation.

   This sounds like an interesting position, and I hope we can talk.
}
